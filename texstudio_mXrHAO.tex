\documentclass[12pt,notitlepage]{article}
\usepackage[utf8]{inputenc}
\usepackage[T2A]{fontenc}
\usepackage[russian]{babel}

\usepackage[left=2cm,right=2cm,top=2cm,bottom=2cm]{geometry}
\setlength\parindent{0pt}

\usepackage{fullpage}
\usepackage{amsmath,amsthm,amssymb}

\DeclareMathOperator{\conv}{conv}
\DeclareMathOperator{\Hom}{Hom}
\DeclareMathOperator{\FF}{FaceFan}
\DeclareMathOperator{\Cone}{Cone}

\newtheorem{theorem}{Теорема}
\newtheorem{definition}{Определение}
\newtheorem{statement}{Утверждение}
\newtheorem{lemma}{Лемма}
\newtheorem*{corollary}{Следствие}

\begin{document}
	\section{Определения и постановка задачи}
	
	Полиэдральным \textbf{конусом}, порождённым векторами $p_1 ... p_m$ в пространстве $R^n$ называется множество $\sum_{i=1}^m \mathbb{R}_{\geq 0} p_i$. Полиэдральный конус называется \textbf{острым}, если не содержит в себе подпространства, большего, чем $\{0\}$ и \textbf{рациональным}, если конечно порождается элементами целочисленной решётки $\mathbb{Z}^n$. Все конусы в $\mathbb{R}^n$ делятся на пары двойственных друг к другу: \textbf{двойственный} конус к $\sigma$ есть конус $\sigma^\nu := \{x \in \mathbb{R}^n | (x, y) \geq 0$ $\forall y \in \sigma\}$. \textbf{Гранью} $\tau \prec \sigma$ конуса $\sigma$ называется $\{x \in \sigma |(x, y) = 0\}$ для некоторого $y \in \sigma^\nu$. Одномерные грани называются \textbf{лучами}. \textbf{Веером} $\Sigma$ называется конечный набор острых рациональных полиэдральных конусов в $R^n$, такой что
	
	1) $\tau \prec \sigma \in \Sigma$ $\Rightarrow$ $\tau \in \Sigma$.
	
	2) $\sigma_1 \in \Sigma$, $\sigma_2 \in \Sigma$ $\Rightarrow$ $\sigma_1 \cap \sigma_2 \prec \sigma_1$, $\sigma_1 \cap \sigma_2 \prec \sigma_2$.
	
	Веер называется \textbf{полным}, если объединение его конусов есть всё $R^n$ и \textbf{билатеральным}, если существует базис целочисленной решётки, каждый вектор которого порождает некоторый луч этого веера, а остальные лучи этого веера лежат в отрицательном ортанте относительно этого базиса. В этом случае этот базис будет порождать один из конусов веера.
	
	Пусть $T$ --- алгебраический тор размерности $n$ с решёткой однопараметрический подгрупп $N \cong \mathbb{Z}^n$ и решёткой характеров $M = \Hom(N, \mathbb{Z}) \cong \mathbb{Z}^n$. Тогда существует взаимно однозначное соответствие между торическими многообразиями и веерами в $N$. .... Более того, существует взаимно однозначное соответствие между полными торическими многообразиями и полными веерами над решёткой однопараметрических подгрупп тора $T$.
	
	Каждый содержащий начало координат выпуклый многогранник $P$ размерности $d$ в $\mathbb{R}^d$ с вершинами в узлах целочисленной решётки соответствует некоторому вееру $\FF(P)$ в $\mathbb{R}^d$, радиус-векторы его вершин являются порождающими векторами лучей веера.
	Два многогранника $P_1$, $P_2$ с вершинами в узлах решётки $\mathbb{Z}^n$ называются \textbf{изоморфными}, если существует невырожденное линейное отображение $\varphi:\mathbb{R}^d \rightarrow \mathbb{R}^n$, такое что $\varphi(\mathbb{Z}^n) = \mathbb{Z}^n$ и $\varphi(P_1) = P_2$.
	
	\textbf{Гладкий $d$-многогранник Фано} --- выпуклый многогранник размерности $d$ в $R^d$ с вершинами в узлах целочисленной решётки, содержащий начало координат, такой что набор радиус-векторов вершин каждой его грани даёт базис целочисленной решётки.
	
	Веера, соответствующие гладким $d$-многогранникам Фано, соответствуют гладким торическим многообразиям Фано, причём классы изоморфности гладких торических многообразий Фано соответствуют классам изоморфности гладких $d$-многогранников Фано.
	
	Полное торическое многообразие $X$  с действием тора $T$ называется \textbf{лучистым}, если максимальная унипотентная подгруппа группы автоморфизмов  многообразия $X$ действует на $X$ с открытой орбитой. Верна следующая теорема:
	
	\begin{theorem}
	Пусть $X$ --- полное торическое многообразие, тогда следующие утверждения эквивалентны:
	
	1) Многообразие $X$ является лучистым;
	
	2) Веер $\Sigma_X$ является билатеральным.
	\end{theorem}
	
	Исследуем лучезарность полных гладких торических многообразий Фано с помощью проверки, являются ли соответствующие веера билатеральными. Полезным будет понятие специального вложения многогранника.
	
	Грань $F$ гладкого $d$-многогранника Фано $P \subset R^d$ называется \textbf{специальной}, если
		\[
		\sum_{v \in V(P)} v = \sum_{v \in V(F)} a_v v, a_v \geq 0
		\] 
	Пусть $P$, $Q$ --- изоморфные гладкие $d$-многогранники Фано, тогда $Q$ называется \textbf{специальныи вложением} для $P$, если $\conv (e_1, ..., e_n)$ является специальной гранью для $Q$.
	
	\section{Связь билатеральности вееров и специальных граней соответствующих многогранников}
	
	У любого гладкого $d$-многогранника Фано есть специальное вложение, так как сумма радиус-векторов его вершин попадёт хотя бы в один конус над гранью этого многогранника, а любая его грань приводится изоморфизмом решётки к стандартному $d-1$-мерному симплексу.
	
	Обозначим ($e_1$, ..., $e_d$) --- стандартный базис $R^d$. Рассмотрим в общем виде гладкий $d$-многогранник Фано $P$, билатеральный относительно некоторой своей специальной грани. В этом разделе представлено доказательство того, что каждая грань многогранника $P$ является его специальной гранью и что многогранник $P$ билатерален относительно каждой своей грани.
	
	\begin{statement}
		Пусть $A$ --- изоморфизм решётки $Z^d$, $\Sigma$ --- веер в пространстве $R^d$. Обозначим его лучи $\rho_1$, ..., $\rho_n$ и их порождающие целочисленные векторы $p_1$, ..., $p_n$ соответственно. Пусть конус $\sigma$ веера $\Sigma$ порождается векторами $p_1$, ..., $p_d$. Тогда веер $\Sigma$ билатерален относительно конуса $\sigma$ тогда и только тогда, когда веер $A\Sigma$ билатерален относительно конуса $A\sigma$.
	\end{statement}
	
	\textbf{Доказательство.} Докажем утверждение слева направо, тогда в обратную сторону будет следовать автоматически применением прямого утверждения для изоморфизма $A^{-1}$, веера $A\Sigma$ с лучами $A\rho_1$, ..., $A\rho_n$ и порождающими векторами $Ap_1$, ..., $Ap_n$ и конусом $A\sigma = <Ap_1, ..., Ap_d>$.
	
	Пусть веер $\Sigma$ билатерален относительно конуса $\sigma$, тогда будем считать, что ($p_1$, ..., $p_d$) есть базис решётки $Z^d$ и $\forall i \in \{d + 1, ..., n\}$ $\exists \alpha_1^i \geq 0$, ..., $\alpha_d^i \geq 0$ $p_i = \alpha_1^i p_1 +$ ... $+ \alpha_d^i p_d$.
	
	Тогда ($Ap_1$, ..., $Ap_d$) есть базис решётки $Z^d$ и $\forall i \in \{d + 1, ..., n\}$ $Ap_i = \alpha_1^i (Ap_1) +$ ... $+ \alpha_d^i (Ap_d)$, значит веер веер $A\Sigma$ билатерален относительно конуса $A\sigma$. \qed
	
	\begin{lemma}
		Пусть $F$ --- специальная грань гладкого $d$-многогранника Фано $P$. Обозначим $p_1$, ..., $p_n$ --- радиус-векторы вершин многогранника $P$, причём концы векторов $p_1$, ..., $p_d$ --- вершины грани $F$. Пусть веер $\FF(P)$ билатерален относительно конуса $\sigma = \Cone$($p_1$, ..., $p_d$). Тогда $\sum_{i = 1}^{n} p_i = 0$.
	\end{lemma}
	
	\textbf{Доказательство.} Рассмотрим веер $\Sigma=\FF(P)$ и обозначим его лучи $\rho_1 = \conv(p_1)$, ..., $\rho_n = \conv(p_n)$. Определим квадратную матрицу $A$, составленную из векторов-столбцов $p_1$, ..., $p_d$. Так как $P$ --- гладкий $d$-многогранник Фано, то $A$ задаёт изоморфизм решётки $Z^d$, значит для него существует обратный изоморфизм целочисленной решётки $A^{-1}$. По доказанному утверждению веер $A^{-1}\Sigma$ билатерален относительно конуса $A^{-1}\sigma = \sum_{i=1}^{d} R_{\geq 0} (A^{-1}p_i) = \sum_{i=1}^{d} R_{\geq 0} e_i$.
	
	Тогда
	\[
	\sum_{i = 1}^{n} p_i = \sum_{i = 1}^{d} a_i p_i, a_i \geq 0
	\]
	\[
	\sum_{i = d + 1}^{n} p_i = \sum_{i = 1}^{d} (a_i - 1) p_i, a_i \geq 0
	\]
	\[
	s := \sum_{i = d + 1}^{n} A^{-1}p_i = \sum_{i = 1}^{d} (a_i - 1) A^{-1}p_i = \sum_{i = 1}^{d} (a_i - 1) e_i, a_i \geq 0 \tag{1} \label{bseeq}
	\]
	Из билатеральности имеем, что каждый вектор $A^{-1}p_i$, $i \in \{d + 1, n\}$ лежит в отрицательном ортанте, тогда (1) даёт, что $s \in \{-1 \leq x_1 \leq 0\} \cap ... \cap \{ -1 \leq x_d \leq 0\}$.
	
	1) $\forall i \in \{1 .. d\}$ $s_i \neq 0$, потому что иначе для некоторого $i$ $P \subset \{s_i \geq 0\}$ и какая-то грань обязана содержать начало координат, что противоречит тому, что радиус-векторы её вершин образуют базис решётки $\mathbb{Z}^d$.
	
	2) $A$ --- изоморфизм решётки $\mathbb{Z}^d$ и координаты вершин многогранника $P$ целочисленные, значит $s$ есть целочисленный вектор.
	
	Таким образом, мы получаем, что $s = (-1, ..., -1)^T$, значит $\forall i \in \{1$, ... $d\}$ $a_i = 0$, то есть $\sum_{i = 1}^{n} p_i = 0$, другими словами, для многогранника $P$ любая его грань является специальной. \qed
	
	\vspace{0.3cm}
	Рассмотрим многогранник $\conv(q_1$, ..., $q_n)$, где $q_i = A^{-1}p_i$. $\forall i \in \{1,$ ..., $d\}$ $q_i = e_i$, а $\forall i \in \{d + 1,$ ..., $n\}$ $q_i$ имеет неположительные целые координаты и $\sum_{i = d + 1}^n q_i = (-1$, ..., $-1)^T$. Итак, этот многогранник обладает следующими свойствами:
	
	\begin{itemize}
		\item $\forall i \in \{1,$ ..., $d\}$ $q_i = e_i$
		
		\item $\forall i \in \{d + 1,$ ..., $n\}$ $\forall j \in \{1,$ ..., $d\}$ $j$-ая координата вектора $q_i$ равна $0$ или $-1$.
		
		\item $\forall i \in \{d + 1,$ ..., $n\}$ $\exists j \in \{1,$ ..., $d\}$ $j$-ая координата вектора $q_i$ равна $-1$.
		
		\item $\forall j \in \{1,$ ..., $d\}$ $\exists ! i \in \{d + 1,$ ..., $n\}$ $j$-ая координата вектора $q_i$ равна $-1$.
	\end{itemize}
	
	Пусть теперь произвольный многогранник $Q$ с вершинами ($q_1$, ..., $q_n$) обладает перечисленными свойствами. Очевидно, что веер, лучи которого соответствуют вершинам этого многогранника, является билатеральным относительно конуса $\Cone(q_1$, ..., $q_d)$. Теперь мы будем доказывать, что $Q$ является гладким $d$-многогранником Фано. Для начала заметим, что эти свойства гарантируют выпуклость: каждая вершина многогранника лежит в некоторой гиперплоскости, относительно которой все остальные вершины лежат по одну сторону. Далее, из третьего и четвёртого свойств следует, что количество его вершин $n$ не превосходит $2d$, а из первого и четвёртого свойств следует, что сумма всех вершин равна нулевому вектору, откуда следует, что начало координат лежит внутри многогранника $Q$. Опишем его $d-1$-мерные грани. Пусть вектор $w$ задаёт грань $F$ многогранника $Q$. Определим число $m$ --- количество таких вершин $q$ многогранника $Q$, что 
	\[
	\langle q, w\rangle = \max_i  \langle q_i, w\rangle.
	\]
	и множество
	\[
	Q_w := \{ i_1 < ... < i_m | \forall k \in \{1, ..., m\} (i_k \in \{1, ..., n\} \wedge \langle q_{i_k}, w\rangle = \max_i  \langle q_i, w\rangle )\},
	\]
	тогда $F=\conv(\{q_i | i \in Q_w\})$. Пусть грань $F$ ($d-1$)-мерна, тогда $m \geq d$. Теперь рассмотрим матрицу $A$, зависящую от $w$, размера $d \times m$, в $j$-ом столбце которой записан вектор $q_{i_j}$.
	
	\begin{statement}
		В каждом столбце, в котором нет элементов $1$ матрицы $A$ есть элемент $-1$, такой что в его строке нет элементов $1$.
	\end{statement}
	
	\textbf{Доказательство.} Предположим противное. Пусть утверждение не выполняется для $j_0$-ого столбца, тогда обозначим индексами $j_1$, ..., $j_l$ номера тех столбцов, в которых есть $1$ в той же строчке, что и для некоторого элемента $-1$ $j_0$-ого столбца. Заметим, что так как в столбцах с номерами $j_1$, ..., $j_l$ записаны векторы стандартного базиса, то других единиц в этих столбцах нет. Составим равную нулю линейную комбинацию вершин $q_{i_1}$, ..., $q_{i_m}$ с неотрицательными коэффициентами, равными в сумме $1$, что даст противоречие с фактом, что начало координат лежит внутри многогранника (а значит не лежит на грани). При каждом векторе вида $q_{i_{j_k}}$, $k \in \{0$, ..., $l\}$ будет коэффициент $\frac{1}{l+1}$, а при остальных векторах будет коэффициент $0$. \qed
	
	\begin{corollary}
		В матрице $A$ столбцов не больше, чем строк.
	\end{corollary}
	
	Таким образом, $m=d$ и каждая грань многогранника $Q$ является выпуклой оболочкой ровно $d$ вершин многогранника. Заметим также ещё одно
	
	\begin{corollary}
		Для каждого столбца без элементов 1 такой выбор элемента -1 единственнен. Другими словами, все остальные элементы -1 каждого столбца, кроме существующего по предыдущему утверждению, имеют в своих строках элемент 1.
	\end{corollary}
	
	Это следует из того, что в одной строке не может быть более одного элемента -1.	Если среди векторов множества $\{q_i | i \in Q_w\}$ ровно $l$ векторов стандартного базиса, то строк без элементов 1 столько же, сколько оставшихся столбцов матрицы, то есть $m-l$. Таким образом среди векторов множества $\{q_i | i \in Q_w\}$ выполняется свойство: каждая координата либо принимает значение 1, либо принимает значение -1, причём в попарно разных векторах. Обозначим $v^k$ для $k \in \{1$, ..., $d\}$ вектор из $\{q_i | i \in Q_w\}$, "отвечающий за $k$-ую координату" в указаном смысле. Для того, чтобы $Q$ был гладким $d$-многогранником Фано, осталось показать, что векторы $\{q_i | i \in Q_w\}$ порождают $\mathbb{Z}^d$. Для этого зададим коэффициенты разложения произвольного вектора $a \in \mathbb{Z}^d$ в линейную комбинацию столбцов матрицы $A$. Если $v^k_k=-1$, положим коэффициент при векторе $v^k$ равным $-a_k$. По последнему следствию уже имеющаяся линейная комбинация будет лежать в $a+\langle\{v^k|v^k_k=1\}\rangle_\mathbb{Z}$, а значит существуют коэффициенты для остальных векторов $v^k$, такие что получившаяся линейная комбинация всех векторов $v_k$ равна $a$. Таким образом, доказана
	
	\begin{lemma}
		Пусть произвольный многогранник $Q$ обладает перечисленными свойствами, тогда $Q$ --- гладкий $d$-многогранник Фано, билатеральный относительно своей специальной грани $\conv(q_1$, ..., $q_d)$.
	\end{lemma}
	
	Наконец, проверим билатеральность многогранника $Q$ относительно его произвольной грани $F$. Пусть вершины грани $F$ записаны в матрицу $A$, как было описано выше, а $a$ --- вершина многогранника $Q$, не лежащая в грани $F$. Предположим, $a$ --- это вектор $e_k$ стандартного базиса. Тогда $v^k_k=-1$ и коэффициент при векторе $v^k$ равен -1. По второму следствию из утверждения 2 все остальные ненулевые коэффициенты появляются при векторах, являющихся элементами стандартного базиса, причём все так же равны -1. Теперь предположим, что $a$ не является вектором стандартного базиса. Так как он не появляется среди столбцов матрицы $A$, то по свойству 4 каждая отрицательная координата вектора $a$ неотрицательна в каждом столбце матрицы $A$, а значит обязательно равна 1 ровно в одном столбце матрицы $A$. Так как все эти столбцы имеют ровно по одной ненулевой координате, их линейная комбинация со всеми коэффициентами равными -1, даёт вектор $a$. Итак, в обоих случаях вектор $a$ выражается через столбцы матрицы $A$ с неположительными коэффициентами, то есть все векторы $a$ лежат в отрицательном ортанте относительно вершин грани $F$.
	
	\section{Классы изоморфности гладких $d$-многогранников Фано, билатеральных относительно своих специальных граней}
	
	Из леммы 2 можно сделать вывод, что классы изоморфности многогранников Фано, билатеральных относительно некоторой своей специальной грани, являются в точности классами изоморфности многогранников, обладающих свойствами 1 --- 4.
	
	Построим отображение $\varphi$ из множества многогранников, удовлетворяющих свойствам 1 --- 4 в множество целочисленных решений уравнения
	\[
	a_1 + ... + a_d = d, \tag{2} \label{bseeq}
	\]
	таких что $d \geq a_1 \geq ... \geq a_d \geq 0$. Для каждого $i \in \{d + 1,$ ..., $n\}$ положим
	\[
	n_i := |\{j \in \{1, ..., d\} : (q_i)_j = -1\}|,
	\]
	тогда образом некоторого многогранника $Q := \conv(q_1$, ..., $q_n)$ под действием отображения $\varphi$ будет набор чисел $n_i$ для $i \in \{d + 1,$ ..., $n\}\}$, дополненный нужным количеством нулей и упорядоченный по невозрастанию.
	
	Очевидно, что многогранники, переходящие под действием $\varphi$ в одно и то же решение уравнения (2), изоморфны: изоморфизм представляет из себя перестановку координат. Кроме того, для любого упорядоченного по невозрастанию целочисленного решения уравнения (2) существует многогранник, переходящий в него под действием $\varphi$. Тогда отображение $\varphi$ индуцирует биекцию между классами изоморфности многогранников со свойствами 1 --- 4 и упорядоченными по невозрастанию целочисленными решениями уравнения (2), если верна следующая лемма.
	
	\begin{lemma}		
		Если многогранники $Q_1$ и $Q_2$ обладают свойствами 1 --- 4 и изоморфны, то $\varphi(Q_1)=\varphi(Q_2)$.
	\end{lemma}
	
	\textbf{Доказательство леммы 3.} При изоморфизме многогранников $Q_1$ и $Q_2$, удовлетворяющих свойствам 1 --- 4, вершины переходят в вершины, а грани в грани. Значит у многогранников совпадают количества вершин и граней, а также для каждой вершины сохраняется количество граней, в которых она лежит. Пусть $\varphi(Q_1) = (a_1 \geq$...$\geq a_d)$, $\varphi(Q_2) = (b_1 \geq$...$\geq b_d)$. Так как количества вершин многогранников равны, то совпадает и наибольший индекс ненулевого элемента среди $\varphi(Q_1)$ и $\varphi(Q_2)$, пусть он равен $k$. Из построения отображения $\varphi$ следует, что существует следующее разбиение множества вершин многогранника $Q_1$ на $k$ подмножеств: $i$-ое подмножество состоит из $a_i + 1$ вершин многогранника $Q_1$ и содержит вершину (обозначим её $q(i)$) с $a_i$ координатами, равными -1, а остальные $a_i$ вершин являются вершинами стандартного симплекса, единица в которых находится в тех координатах, в которых находится -1 у вершины $q(i)$. Количество граней многогранника $Q_1$, как и количество граней многогранника $Q_2$, равно
	\[
	(a_1+1)\cdot ... \cdot (a_k+1)=(b_1+1)\cdot ... \cdot (b_k+1), \tag{3} \label{bseeq}
	\]
	так как в произвольной грани многогранника $Q_1$  лежит ровно $a_i$ вершин из $i$-ого подмножества разбиения (аналогичное верно и для $Q_2$).
	
	Будем обозначать $V(Q)$ множество вершин многогранника $Q$. Для $q \in V(Q_i)$ ($i \in \{1, 2\}$) определим $F_i(q)$ как количество граней многогранника $Q_i$, содержащих $q$. Для $i \in \{1, 2\}$ определим мультимножество $M_i = \{F_i(q)| q \in V(Q_i)\}$, тогда $M_1 = M_2$. Если вершина $q \in V(Q_1)$ лежит в $i$-ом подмножестве разбиения, то
	\[
	F_1(q)=f_1(i):=(a_1+1)\cdot ... \cdot (a_{i-1}+1)\cdot a_i\cdot (a_{i+1}+1)\cdot ... \cdot (a_k+1).
	\]
	Аналогично определяется и $f_2(i)$.
	
	Таким образом, $M_1$ состоит из дизъюнктного объединения по $i \in \{1$, ..., $k\}$ мультимножеств из $a_i + 1$ одинаковых чисел $f_1(i)$. Аналогично, $M_2$ состоит из дизъюнктного объединения по $i \in \{1$, ..., $k\}$ мультимножеств из $b_i + 1$ одинаковых чисел $f_2(i)$. Теперь можно считать, что $M_1$ и $M_2$ можно задать, имея исключительно наборы чисел $(a_1 \geq$...$\geq a_k)$ и $(b_1 \geq$...$\geq b_k)$. Докажем следующее вспомогательное утверждение индукцией по $k$:
	
	Пусть даны наборы чисел $a_1 \geq ... \geq a_k \geq 1$ и $b_1 \geq ... \geq b_k \geq 1$, для которых выполняется равенство (3) и $M_1 = M_2$, тогда $\forall i \in \{1$, ..., $k\}$ $a_i=b_i$.
	
	База($k=1$): $a_1+1=b_1+1$ $\Rightarrow$ $a_1=b_1$.
	
	Переход($k>1$): $\forall i < j$ $a_i \geq a_j$. Домножим неравенство на левую часть равенства (3), поделим на $(a_i+1)(a_j+1)$ и получим $f_1(i) \geq f_1(j)$. Аналогично, $\forall i < j$ $f_2(i) \geq f_2(j)$. Значит, в множестве $M_1$ максимальный элемент равен $f_1(1)$, а в множестве $M_2$ максимальный элемент равен $f_2(1)$, следовательно $f_1(1) = f_2(1)$. Поделим на равенство (3) и получим
	\[
	\frac{a_1}{a_1+1}=\frac{b_1}{b_1+1}
	\]
	Значит, $a_1=b_1$. Если убрать из $M_1$ $a_1+1$ максимальных элементов, получится мультимножество $M'_1$, строящееся по набору $(a_2 \geq$...$\geq a_k)$, а если убрать из $M_2$ $b_1+1$ максимальных элементов, получится мультимножество $M'_2$, строящееся по набору $(b_2 \geq$...$\geq b_k)$, они также будут совпадать. Кроме того, будет выполняться
	\[
	(a_2+1)\cdot ... \cdot (a_k+1)=(b_2+1)\cdot ... \cdot (b_k+1),
	\]
	значит, по предположению индукции $\forall i \in \{2$, ..., $k\}$ $a_i=b_i$, откуда следует переход индукции. Вспомогательное утверждение доказано.
	
	По вспомогательному утверждению получаем, что $\forall i \in \{1$, ..., $k\}$ $a_i=b_i$, следовательно, $\varphi(Q_1)=\varphi(Q_2)$.
	\qed
	 
	Таким образом, если $P$ --- гладкий $d$-многогранник Фано, билатеральный относительно некоторой своей специальной грани, то $P$ принадлежит одному из $T(d)$ классов изоморфности, где $T(d)$ --- количество неотрицательных целых решений уравнения $x_1 + ... + x_d = d$, таких что $d \geq x_1 \geq ... \geq x_d \geq 0$.
\end{document}
