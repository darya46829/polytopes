\documentclass[12pt,notitlepage]{article}
\usepackage[utf8]{inputenc}
\usepackage[T2A]{fontenc}
\usepackage[russian]{babel} 

\usepackage{fullpage}
\usepackage{amsmath,amsthm,amssymb}

\DeclareMathOperator{\conv}{conv}
\DeclareMathOperator{\Hom}{Hom}
\DeclareMathOperator{\FF}{FaceFan}
\DeclareMathOperator{\Cone}{Cone}

\newtheorem{theorem}{Теорема}
\newtheorem{definition}{Определение}
\newtheorem{statement}{Утверждение}

\begin{document}
	
	Полиэдральный \textbf{конус}, порождённый векторами $e_1 ... e_m$ в пространстве $R^n$ --- это $\sum_{i=1}^m R_{\geq 0} e_i$. Полиэдральный конус называется \textbf{острым}, если не содержит в себе подпространства, большего, чем $\{0\}$ и \textbf{рациональным}, если конечно порождается элементами целочисленной решётки $Z^n$. Все конусы в $R^n$ делятся на пары двойственных друг к другу: \textbf{двойственный} конус к $\sigma$ есть конус $\sigma^\nu := \{x \in R^n | (x, y) \geq 0$ $\forall y \in \sigma\}$. \textbf{Гранью} $\tau \prec \sigma$ конуса $\sigma$ называется $\{x \in \sigma |(x, y) = 0\}$ для некоторого $y \in \sigma^\nu$. Одномерные грани называются \textbf{лучами}. \textbf{Веером} $\Sigma$ называется конечный набор острых рациональных полиэдральных конусов в $R^n$, такой что
	
	1) $\tau \prec \sigma \in \Sigma$ $\Rightarrow$ $\tau \in \Sigma$.
	
	2) $\sigma_1 \in \Sigma$, $\sigma_2 \in \Sigma$ $\Rightarrow$ $\sigma_1 \cap \sigma_2 \prec \sigma_1$, $\sigma_1 \cap \sigma_2 \prec \sigma_2$.
	
	Веер называется \textbf{полным}, если объединение его конусов есть всё $R^n$ и \textbf{билатеральным}, если существует базис целочисленной решётки, каждый вектор которого порождает некоторый луч этого веера, а остальные лучи этого веера лежат в отрицательном ортанте относительно этого базиса. В этом случае этот базис будет порождать один из конусов веера.
	
	Пусть $T$ --- алгебраический тор размерности $n$ с решёткой однопараметрический подгрупп $N \cong Z^n$ и решёткой характеров $M = \Hom(N, Z) \cong Z^n$. Тогда существует взаимно однозначное соответствие между торическими многообразиями и веерами в $N$. .... Более того, существует взаимно однозначное соответствие между полными торическими многообразиями и полными веерами над решёткой однопараметрических подгрупп тора $T$.
	
	Каждый содержащий начало координат выпуклый многогранник $P$ размерности $d$ в $R^d$ с вершинами в узлах целочисленной решётки соответствует некоторому вееру $\FF(P)$ в $R^d$, его вершины являются порождающими лучей веера.
	Два многогранника $P_1$, $P_2$ с вершинами в узлах решётки $Z^n$ называются \textbf{изоморфными}, если существует биекция $\varphi:R^d \rightarrow R^n$, такая что $\varphi(Z^n) = Z^n$ и $\varphi(P_1) = P_2$.
	
	\textbf{Гладкий $d$-многогранник Фано} --- выпуклый многогранник размерности $d$ в $R^d$ с вершинами в узлах целочисленной решётки, содержащий начало координат, такой что набор вершин каждой его грани даёт базис целочисленной решётки.
	
	Веера, соответствующие гладким $d$-многогранникам Фано, соответствуют гладким торическим многообразиям Фано, причём классы изоморфности гладких торических многообразий Фано соответствуют классам изоморфности гладких $d$-многогранников Фано.
	
	Полное торическое многообразие $X$  с действием тора $T$ называется \textbf{лучистым}, если максимальная унипотентная подгруппа группы автоморфизмов  многообразия $X$ действует на $X$ с открытой орбитой.
	
	\begin{theorem}
	Пусть $X$ --- полное торическое многообразие, тогда следующие утверждения эквивалентны:
	
	1) Многообразие $X$ является лучистым;
	
	2) Веер $\Sigma_X$ является билатеральным.
	\end{theorem}
	
	Исследуем лучистость полных гладких торических многообразий Фано с помощью проверки, являются ли соответствующие веера билатеральными. Полезным будет понятие специального вложения многогранника.
	
	Грань $F$ многогранника $P \subset R^d$ называется \textbf{специальной}, если
		\[
		\sum_{v \in V(P)} v = \sum_{v \in V(F)} a_v v, a_v \geq 0
		\] 
	Пусть $P$, $Q$ --- изоморфные гладкие $d$-многогранники Фано, тогда $Q$ называется \textbf{специальныи вложением} для $P$, если $\conv (e_1, ..., e_n)$ является специальной гранью для $Q$.
	
	У любого гладкого $d$-многогранника Фано есть особое вложение, так как сумма радиус-векторов его вершин попадёт хотя бы в один конус над гранью этого многогранника, а любая его грань является специальной.
	
	Пусть $(e_1, ... e_d)$ --- стандартный базис пространства $R^d$, $A$ --- изоморфизм пространства $R^d$, переводящий порождающие векторы некоторых $d$ лучей веера $\Sigma$ в $\{e_i | i \in \{1 .. d\}\}$, тогда все линейные зависимости между порождающими векторами лучей веера $\Sigma$ сохраняются для их образов, то есть порождающих векторов лучей веера $A \Sigma$, также в обратную сторону они тоже сохраняются, так как $A^{-1}$ --- тоже изоморфизм пространства. Значит билатеральность веера $\Sigma$ относительно своего конуса $\sigma$ равносильна билатеральности относительно конуса $<e_1, ..., e_d>_{R_{\geq 0}}$ веера, лучи котороого порождаются векторами $\{A v_i | <v_i>_{R_{\geq 0}}$ --- луч веера $\Sigma \}$, где $A$ --- матрица, обратная к матрице, столбцы которой образуют набор порождающих векторов конуса $\sigma$.
	
	Пусть $P$ --- выпуклый многогранник в $R^d$ максимальной размерности, содержащий начало координат. Предположим, что у него есть специальная грань $F$ и к тому же веер $\Sigma=\FF(P)$ билатерален относительно конуса $\Cone(F)$. Обозначим $A$ --- изоморфизм пространства $R^d$, переводящий радиус-векторы вершин грани $F$ в $\{e_i | i \in \{1 .. d\}\}$
	
	Тогда
		\[
		\sum_{v \in V(P)} v = \sum_{v \in V(F)} a_v v, a_v \geq 0
		\]
		\[
		\sum_{v \in V(P)\textbackslash V(F)} v = \sum_{v \in V(F)} (a_v - 1) v, a_v \geq 0
		\]
		\[
		\sum_{v \in V(P)\textbackslash V(F)} Av = \sum_{v \in V(F)} (a_v - 1) Av, a_v \geq 0
		\]
	Из билатеральности имеем, что каждый вектор, входящий в сумму в левой части, лежит в отрицательном ортанте, тогда последнее равенство даёт, что значение $s = (s_1, ... s_d) \in R^d$ сумм в левой и правой частях лежит в кубе $\{-1 \leq x_1 \leq 0\} \cap ... \cap \{ -1 \leq x_d \leq 0\}$.
	
	Теперь предположим, что $P$ --- это гладкий $d$-многогранник Фано, тогда
	
	1) $\forall i \in \{1 .. d\}$ $s_i \neq 0$, потому что иначе для некоторого $i$ $P \subset \{s_i \geq 0\}$ и какая-то грань обязаны содержать начало координат, что противоречит тому, что радиус-векторы её вершин образуют базис решётки $Z^d$.
	
	2) $A$ --- изоморфизм решётки $Z^d$ и координаты вершин многогранника $P$ целочисленные, значит левая часть равенства есть целочисленный вектор.
	
	Таким образом, мы получаем, что $\forall i \in \{1 .. d\}$ $s_i = -1$ и из этого следует
	
	1) $\forall v \in V(F)$ $a_v = 0$, то есть $\sum_{v \in V(P)} v=0$, значит для многогранника $P$ любая его грань является специальной.
	
	2) Обозначим полученный из $P$ с помощью изоморфизма $A$ многогранник $Q$, тогда помимо вершин стандартного $d-1$-мерного симплекса $Q$ имеет только вершины, координаты которых есть наборы из $0$ и $-1$, причём индексы отрицательных координат, дизъюнктно объединяясь, дают множество $\{1 .. d\}$.
	
	Тогда $P$ принадлежит одному из $K(d)$ классов изоморфности, где $K(d)$ --- количество неотрицательных целых решений уравнения $x_1 + ... + x_d = d$, таких что $n \geq x_1 \geq ... \geq x_d \geq 0$.
	
	Если доказать, что многогранник, являющийся выпуклой оболочкой концов стандартного базиса решётки $Z^d$ и точек с равными $-1$ координатами с номерами от $1$ до $x_1$, от $x_1 + 1$ до $x_1 + x_2$, ..., от $x_1 + ... + x_{d-1} + 1$ до $d$ и нулевыми всеми остальными координатами для каждого такого решения уравнения, будет многогранником Фано, то классов изоморфности гладких $d$-многогранников Фано со специальной гранью, относительно которой соответствующий веер билатерален, ровно $K(n)$.
	
	Какие будут грани $Q$ и будет ли веер $\FF(Q)$ билатерален относительно конуса над каждой из них?
	
\end{document}
