\documentclass[12pt,notitlepage]{article}
\usepackage[utf8]{inputenc}
\usepackage[T2A]{fontenc}
\usepackage[russian]{babel} 

\usepackage{fullpage}
\usepackage{amsmath,amsthm,amssymb}

\DeclareMathOperator{\conv}{conv}

\newtheorem{theorem}{Теорема}
\newtheorem{definition}{Определение}
\newtheorem{statement}{Утверждение}

\begin{document}
	\section{Необходимые сведения}
	
	\begin{definition}
		(Полиэдральный) \textbf{конус}, порождённый векторами $e_1 ... e_m$ в пространстве $R^n$ --- это $\sum_{i=1}^m R_{\geq 0} e_i$.
		
		Называется \textbf{острым}, если не содержит в себе подпространства, большего, чем $\{0\}$.
		
		Называется \textbf{рациональным}, если конечно порождается элементами целочисленной решётки.
	\end{definition}
	\begin{statement}
		Все конусы в $R^n$ делятся на пары двойственных друг к другу: \textbf{двойственный} конус к $\sigma$ есть конус $\sigma^\nu := \{x \in R^n | (x, y) \geq 0 \forall y \in \sigma\}$.
	\end{statement}
	\begin{definition}
		\textbf{Гранью} $\tau \prec \sigma$ конуса $\sigma$ называется $\{x \in \sigma |(x, y) = 0\}$ для некоторого $y \in \sigma^\nu$. Одномерные грани называются \textbf{лучами}.
	\end{definition}
	\begin{definition}
		\textbf{Веером} $\Sigma$ называется конечный набор острых рациональных полиэдральных конусов в $R^n$, такой что
		
		1) $\tau \prec \sigma \in \Sigma$ $\Rightarrow$ $\tau \in \Sigma$.
		
		2) $\sigma_1 \in \Sigma$, $\sigma_2 \in \Sigma$ $\Rightarrow$ $\sigma_1 \cap \sigma_2 \prec \sigma_1$, $\sigma_1 \cap \sigma_2 \prec \sigma_2$.
		
		Называется \textbf{полным}, если объединение его конусов есть всё $R^n$.
		
		Называется \textbf{двусторонним (bilateral)}, если существует базис целочисленной решётки, каждый вектор которого порождает некоторый луч этого веера, а остальные лучи этого веера лежат в отрицательном ортанте относительно этого базиса.
	\end{definition}
	\begin{definition}
		\textbf{Гладкий $d$-многогранник Фано} --- содержащий начало координат выпуклый многогранник размерности $d$ в $R^d$ с вершинами в узлах целочисленной решётки, такой что набор вершин каждой его грани даёт базис целочисленной решётки.
	\end{definition}
	\begin{definition}
		Два гладких многогранника Фано $P_1$, $P_2$ называются \textbf{изоморфными}, если существует биекция $\varphi:R^d \rightarrow R^d$, такая что $\varphi(Z^d) = Z^d$ и $\varphi(P_1) = P_2$.
	\end{definition}
	\begin{statement}
		Каждый гладкий $d$-многогранник Фано соответствует некоторому вееру в $R^d$, их вершины являются порождающими лучей веера.
		
		Разные многогранники Фано всегда соответствуют разным веерам (так как либо отличается набор лучей, либо набор конусов с теми же лучами).(?)
	\end{statement}
	
	\section{Мотивация}
	
	\begin{definition}
		Полное торическое многообразие $X$  с действием тора $T$ называется \textbf{лучистым (radiant)}, если максимальная унипотентная подгруппа группы автоморфизмов  многообразия $X$ действует на $X$ с открытой орбитой.
	\end{definition}
	\begin{theorem}
		Существует взаимно однозначное соответствие между полными торическими многообразиями и полными веерами над решёткой однопараметрических подгрупп тора $T$.
		
		Соответствующий многообразию $X$ веер обозначается $\Sigma_X$.
	\end{theorem}
	\begin{theorem}
		Пусть $X$ --- полное торическое многообразие, тогда следующие утверждения эквивалентны:
		
		1) Многообразие $X$ является лучистым;
		
		2) Веер $\Sigma_X$ является двусторонним.
	\end{theorem}
	
	\begin{theorem}
		Классы изоморфности гладких торических многообразий Фано размерности $d$ соответствуют классам изоморфности гладких $d$-многогранников Фано.
	\end{theorem}
	
	\section{Формулировка задачи}
	
	Существуют списки гладких $d$-многогранников Фано для $d = 3, 4, ...$. Исследуем лучистость полных гладких торических многообразий Фано с помощью проверки, являются ли соответствующие веера двусторонними. Полезным будет понятие особого вложения многогранника.
	
	\begin{definition}
		Грань $F$ гладкого $d$-многогранника Фано $P$ называется \textbf{особой}, если
		\[
		\sum_{v \in V(P)} v = \sum_{w \in V(F)} a_w w, a_w \geq 0
		\]
	\end{definition} 
	\begin{definition}
		Пусть $P$, $Q$ --- изоморфные гладкие $d$-многогранники Фано, тогда $Q$ называется \textbf{особым вложением} для $P$, если $\conv (e_1, ..., e_n)$ является особой гранью для $Q$.
	\end{definition}
	
	У любого гладкого $d$-многогранника Фано есть особое вложение. Обозначим $\Sigma_P$ веер, соответствующий гладкому $d$-многограннику Фано $P$.
	
	\begin{statement}(?)
		Веер, соответствующий гладкому $d$-многограннику Фано, двусторонний тогда и только тогда, когда веер, соответствующий некоторому (не обязательно любому) его особому вложению, является двусторонним относительно базиса $e_1, ..., e_n$.
	\end{statement}
	
	Таким образом, если $X$ --- полное гладкое торическое многообразие Фано, то $X$ лучистое тогда и только тогда, когда веер $\Sigma_P$ двусторонний относительно базиса $e_1, ..., e_n$ для некогорого особого вложения $P$ из соответствующего класса изоморфонсти гладких $d$-многогранников Фано.
\end{document}